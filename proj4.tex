\documentclass[a4paper, 11pt]{article}
\usepackage[czech]{babel}
\usepackage[utf8]{inputenc}
\usepackage[left=2cm, top=3cm, text={17cm, 24cm}]{geometry}
\usepackage{times}
\usepackage[unicode]{hyperref}

\begin{document}


\begin{titlepage}
\begin{center}

\huge
{FAKULTA INFORMAČNÍCH TECHNOLOGÍ\\[0.4em]
VYSOKÉ UČENÍ TECHNICKÉ V BRNĚ}\\
\vspace{\stretch{0.382}}
\LARGE
{Typografie a publikování \,--\, 4. projekt \\[0.3em] 
\textbf{\huge{Bibliografické citace}}}
\vspace{\stretch{0.618}}

{\LARGE \today \hfill 
{Matej Mištík }}

\end{center}
\end{titlepage}

	\section{Systém \LaTeX}
	
	\subsection{Pro koho je latex}
	Je vybraný pro zkušené  uživatele, kteří potřebují publikace světové úrovni ,víc v~knize \cite{Knuth1990}, ale aj pro laiky použitím online \LaTeX\ editoru, například \cite{Simecek2013}.Hlavními výhodami vůči ostatním editorům jsou tedy 
		\begin{enumerate}
		\item GUI editoru googlu,
		\item vykreslování výsledku do okna prohlížeče přes obrazové soubory,
		\item zadávání parametrů pro překlad,
		\item kooperace více lidí.
	\end{enumerate}
	Podrobnější informace o východech lze dočíst v~\cite{Sokol2012}.

	
	\subsection{Historie}
	Autorem je Donald Ervin Knuth a vytvořil \TeX\ proto, aby mohl své odborné texty publikovat v odpovídající kvalitě. Protože sazeči pracující v tiskárně bez matematického vzdělání, tvořili moc chyb při sázení vzorců.
    \TeX\ poskytuje mechanismus pro definici nových příkazů. Proto byla vytvořena
    celá řada nadstaveb, které usnadnily práci s velice složitým jazykem. jednou z nejznámějších je \LaTeX\ .Více zjistíte na \cite{Pysny2009}.
    Aby smě předšili historii, tak musíme vyslovovat slovo \TeX\ správné, a~to následovně "tek", víc o historii slova se dozvíte zde \cite{Slantchev}
    
    
    \subsection{Složení dokumentu}
    
    Každí \TeX dokument se skládá z preambuli a těla dokumentu.Více o těchto částech se dočtete v knize Guide to \LaTeX\ \cite{Kopka2004}.
    
    \subsection{Začátky a první dokument}
    Mohlo by se zdát že, \LaTeX\ je nemožné pře smrtelníka se naučit.Tak tvrdí článek Veselého, vysvětlení jak na to najdeme v článku. \cite{Vesely1991}
    
    
    \subsubsection{Titulní strana}
    \verb|\maketitle| je příkaz pro vytvoření titulní stránky - tam je název, autor atd.. V případě volby notitlepage v~\verb|\documentclass| (implicitní pro třídu article) je tato \uv{strana} umístěna v~dokumentu v~horní části stránky na níž pak normálně pokračuje další text a pro tuto stránku je použit styl plain (\verb|\thispagestyle{plain}|).\\ Další možnosti najdeme zde \cite{Sopouch2001}.
    
    
    \subsection{Nové symboly STIX MATH}
    
    V novém se podařilo najít zkombinováním symbolů dalších 2000 matematických symbolů. víc v článku \cite{GratzerG2015}
    
    \section{Matematika}
    
    \subsection{Matematika dětí}
    
    Dle výzkumu 578 prváků, počítaní doma s rodiči výrazně zvyšuje jejich úspěšnost v skole. víc v \cite{BerkowitzTalia2015}
    
    
    
    
    
    
    
    
    
    
    
    
   
    
    
	

	
	
	
	\newpage
	\bibliographystyle{czechiso}
	\renewcommand{\refname}{Použitá Literatura}
	\bibliography{proj4}


\end{document}
